\documentclass{jsarticle}
\begin{document}

\title{Mimblewimble and Grin(メモ)}
\author{じょんた}
\maketitle


\section{Introduction}
\begin{itemize}
    \item GrinはMimblewimbleを実装するためのプロジェクトの名前
    \item デフォルトで匿名性を保とうぜ
    \item トランザクションの数ではなく、ユーザーの数でスケールする。したがって従来のブロックチェーンと比べて容量を節約できる
    \item 楕円曲線暗号を使う
    \item asic-resistentというマイニングアルゴリズムがdecentralization をencourageするらしい
\end{itemize}

\section{Tongue Tying for Everyone}

\subsection{ECCについて簡単に}
楕円曲線上の点の集合について考える。ある点$H$について、和と積が定義されていて、それらは交換法則と結合法則を満たす。ある数$k$を
点$H$に掛けた答え$k*H$もまた、楕円曲線上の点の集合に属する。また、実数$k, j$と点$H$の計算には分配法則も成り立つ。

\begin{equation}\label{bunpai}
    (k+j)*H = k*H + j*H
\end{equation}

楕円曲線暗号において考えると、十分に大きい数$k$を秘密鍵とすると、それに$H$を掛けた$k*H$は公開鍵に対応するものだと考えられる。暗号をとこうとして$k*H$の値から$k$の値を
計算することは不可能に近いので(EC上での「割り算」はとてもむずかしい)、秘密鍵が悪意のある第三者に知られることはない。

式(\ref{bunpai})が示すのは、2つの秘密鍵の和から計算された公開鍵は、それぞれの秘密鍵から計算された公開鍵の和に等しいということ。BitcoinのHDウォレット(階層決定ウォレット?)
やMimblewimbleの実装は、この原理が出発点らしい。(Mimblewimbleのは点を合計するとゼロってやつだと思われる)

\subsection{MimblewimbleでのTransaction}
Mimblewimbleでのトランザクションの承認は以下の2つの原理によって行われる
\begin{enumerate}
    \item input-outputの値がゼロに等しいこと。これだけを確認すればいいので実際のinputとoutputの量は不必要
    \item 秘密鍵を所有していることの証明:これは他のブロックチェーンと違うのは、直接トランザクションに署名をするわけではない
\end{enumerate}

\subsection{残高}
トランザクションの値とは何か。

$v$がトランザクションinputまたはoutputだとすると、$v$の代わりに$v*H$をトランザクションに組み込める。また、それらの値の和についても
\begin{equation}
    v_1 + v_2 = v_3 => v_1*H + v_2*H = v_3*H
\end{equation}

ということが分かる。

\begin{center}
	commitment = SHA256( (blinding factor) or (data) )
\end{center}


\end{document}


